\documentclass{article}

\usepackage[a4paper,top=2cm,bottom=2cm]{geometry}
\usepackage[utf8]{inputenc}
\usepackage{hyperref}
\usepackage[english]{babel}
\usepackage{float}
\usepackage{pgfgantt}
\usepackage{multicol}

\hypersetup{pdfinfo={
  Author={Andrii Nemchenko},
  Title={Curriculum Vitae}
}}

\hypersetup{pdfinfo={
  Author={Andrii Nemchenko},
  Title={Curriculum Vitae}
}}
\begin{document}
% HEVEA \begin{divstyle}{container}
\title{Curriculum Vitae}
\author{
  Andrii Nemchenko\\
  telegram: \href{https://t.me/tellnobody1}{@tellnobody1}\\
  email: \href{mailto:a.nemchenko@icloud.com}{a.nemchenko@icloud.com}\\
  location: Kyiv, Ukraine
}
\maketitle
\section{Experience}
\subsection{Web Developer (Scala, PureScript)}
6 years (Apr 2013 – Present)\\
Playtech
\paragraph{Description}
Developing and evolving Web Platform for licensees to create portals with player's account management, games hub, content-management system (CMS) and integration with 3rd-party services and data providers (AS).
\subsection{Frontend Developer (Java)}
1 year (Apr 2012 – Mar 2013)\\Playtech
\paragraph{Description}
Developing cross-platform mobile games (HTML5 Canvas, iOS, Android) using modified Google PlayN framework to meet performance expectations for players. Use CSS 3 for native 3D HTML games. Creating a common framework for slot games to eliminate code duplication and deliver large number of titles in tight schedule.
\subsection{Web Developer (PHP)}
2 years (from Apr 2010 – Mar 2012)
\paragraph{Description}
Writing not so big websites and also e-commerce, survey analytics platform, social networks for local companies, corporations and startups. Diving into PHP, MySQL, using some CSS, JavaScript, looking at Photoshop.
\section{Education}
\subsection{Kyiv National Taras Shevchenko University}
Master of Science (2006 – 2012)\\
Applied mathematics
\section{Certification}
\subsection{Functional Programming Principles in Scala}
EPFL (École polytechnique fédérale de Lausanne)\\
Dec 2013
\section{Projects}
\subsection{KVS}
Jul 2014 – Present\\Playtech
\paragraph{Stack} Scala, Akka (cluster, cluster-sharding), LevelDB (via jnr-ffi)
\paragraph{Description}
Closed-sourced abstract Scala storage framework with high-level API for handling linked lists of polymorphic data (feeds).\\
Designed with various backends in mind and to work in pure JVM environment. Implementation based on top of KAI (implementation of Amazon DynamoDB in Erlang) port with modification to use akka-cluster infrastructure.\\
Currently main backend is LevelDB to support embedded setup alongside application. Feed API (add/entries/remove) is built on top of Key-Value API (put/get/delete).\\
KVS is highly available distributed (AP) strong eventual consistent (SEC) and sequentially consistent (via cluster sharding) storage. It is used for data from sport and games events. In some configurations used as distributed network file system. Also can be a generic storage for application.
\subsection{CMS}
Apr 2013 – Present\\Playtech
\paragraph{Stack} Scala, Akka (cluster-sharding, http-core), PureScript, Scala.js, Scalaz (core, effect), Argonaut, Scodec, Protobuf
\paragraph{Description}
Content management system with one cluster setup to serve dozen licensees each with up to 10 websites. CMS is all about storing data and working with UI. Data is stored to distributed KVS and UI is built with pure functional strongly typed language (PureScript) which produces robust and fail-safe UI. User files are handled by distributed filesystem and meta data is saved to KVS.
\subsection{AS}
Jul 2014 – Feb 2017\\Playtech
\paragraph{Stack} Scala, Akka (actor, cluster, stream, http-core), Cats (effect), Scalaz (core), Argonaut, Pickling
\paragraph{Description}
Application server with streaming idea in its core. The integration layer for services providers which unifies the different APIs and respect the providers limitations guarding their services from unexpected usage. Unified services structure with akka-stream based IO layer and KVS distributed storage engine. Also plays as the service provider for itself to provide event streaming for sports betting and additional data store interfaces for the client application.
\subsection{Monitoring}
Feb 2015 – Present\\Playtech
\paragraph{Stack} Scala, Akka (stream), PureScript, React.js, UDP, JMX
\paragraph{Description}
Monitoring application receives the metrics, stores and display them in the dashboard. With some basic analyse and triggers the monitor can notify the interested parties in the specific events (application crash, node down, specific user event occurs, etc). Visualization and performance reporting.
\subsection{Documentation}
\paragraph{Stack} Scala, Akka (http-core), LuaLaTeX, Hevea
\paragraph{Description}
Documenting system and server for the applications. LaTeX based tools for generating the project documentation in HTML and PDF format.
\subsection{Open API}
\paragraph{Stack} Java, OSGi
\paragraph{Description}
The platform provides the components to build the services which will be available to clients through unified Open API. The Framework is the set of applications which are bundled together as an OSGi features and OSGi bundles.
\section{Interests}
Blockchain, Compilers, Navigation\\
Idris, Haskell, PureScript, Erlang, Rust, Swift, Kotlin\\
HoTT, CT, Coq, Quantum computing
% HEVEA \end{divstyle}
\end{document}
