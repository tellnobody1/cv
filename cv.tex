\documentclass{article}

\usepackage[a4paper,top=2cm,bottom=2cm]{geometry}
\usepackage[utf8]{inputenc}
\usepackage{hyperref}
\usepackage[english]{babel}
\usepackage{float}
\usepackage{pgfgantt}
\usepackage{multicol}

\hypersetup{pdfinfo={
  Author={Andrii Nemchenko},
  Title={Curriculum Vitae}
}}

\begin{document}
% HEVEA \begin{divstyle}{container}
\title{Curriculum Vitae}
\author{
  Andrii Nemchenko\\
  telegram: \href{https://t.me/tellnobody1}{\nolinkurl{@tellnobody1}}\\
  email: \mailto{a.nemchenko@icloud.com}\\
  github: \href{https://github.com/tellnobody1}{\nolinkurl{@tellnobody1}}\\
  location: Kyiv, Ukraine
}
\maketitle

\section{Experience}
\subsection{Web Scala/PureScript Developer}
5 years (Jul 2014 – Present)\\Playtech (PTEC)
\paragraph{Description}
Developing and evolving Web Platform for licensees to create portals with player's account management, games hub, content-management system (CMS) and integration with 3rd-party services and data providers (AS).

\subsection{Web Java Developer}
1 year

\subsection{Mobile Java Developer}
1 year

\subsection{Web PHP/JavaScript Developer}
2 years

\section{Education}
\paragraph{Kyiv National Taras Shevchenko University}
Master of Science (Applied mathematics)

\section{Certification}
\paragraph{Functional Programming Principles in Scala}
EPFL (École polytechnique fédérale de Lausanne)

\section{Projects}
\subsection{KVS}
Jul 2014 – Present\\
Playtech\\
\footahref{https://github.com/zero-deps/kvs}{zero-deps/kvs}
\paragraph{Stack} Scala, Akka Cluster, LevelDB
\paragraph{Description}
Abstract Scala storage framework with high-level API for handling linked lists of polymorphic data (feeds).\\
Designed with various backends in mind and to work in pure JVM environment. Implementation based on top of KAI (implementation of Amazon DynamoDB in Erlang) port with modification to use akka-cluster infrastructure.\\
Currently main backend is LevelDB to support embedded setup alongside application. Feed API (add/entries/remove) is built on top of Key-Value API (put/get/delete).\\
KVS is highly available distributed (AP) strong eventual consistent (SEC) and sequentially consistent (via cluster sharding) storage. It is used for data from sport and games events. In some configurations used as distributed network file system. Also can be a generic storage for application.

\subsection{CMS}
Feb 2017 – Oct 2019\\Playtech
\paragraph{Stack} Scala, Akka Cluster, PureScript, Protobuf
\paragraph{Description}
Content management system with one cluster setup to serve dozen licensees each with up to 10 websites. CMS is all about storing data and working with UI. Data is stored to distributed KVS and UI is built with pure functional strongly typed language (PureScript) which produces robust and fail-safe UI. User files are handled by distributed filesystem and meta data is saved to KVS.

\subsection{AS}
Jul 2014 – Feb 2017\\Playtech
\paragraph{Stack} Scala, Akka Cluster, Akka Streams
\paragraph{Description}
Application server with streaming idea in its core. The integration layer for services providers which unifies the different APIs and respect the providers limitations guarding their services from unexpected usage. Unified services structure with akka-stream based IO layer and KVS distributed storage engine. Also plays as the service provider for itself to provide event streaming for sports betting and additional data store interfaces for the client application.

\subsection{Monitoring}
Feb 2015 – Present\\Playtech
\paragraph{Stack} Scala, Akka Streams, PureScript
\paragraph{Description}
Monitoring application receives the metrics, stores and display them in the dashboard. With some basic analyse and triggers the monitor can notify the interested parties in the specific events (application crash, node down, specific user event occurs, etc). Visualization and performance reporting.

\subsection{Documentation}
Playtech
\paragraph{Stack} Scala, LaTeX, HeVeA
\paragraph{Description}
Documenting system and server for the applications. LaTeX-based tools for generating the project documentation in HTML and PDF format.

\section{Interests}
Blockchain | Compiler | Navigation\\
Idris | Haskell | Dotty | PureScript | Erlang | Rust | Swift | Kotlin\\
HoTT | CT | Coq | Quantum Computing\\
Longboarding | Snowboarding | Travelling\\
Animation | Music | Books\\
Boardgames | PS4
% HEVEA \end{divstyle}
\end{document}